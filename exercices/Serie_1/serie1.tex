%
%
%
%

\documentclass{article}
         
\usepackage{graphicx,xcolor}
\usepackage[load-headings]{exsheets}
%\usepackage{bbding, skull}
\usepackage[frenchb]{babel}
\usepackage{amsmath,amssymb,bm}
\usepackage{enumitem}
\usepackage{lmodern,microtype}
\usepackage[a4paper,left=4cm,right=4cm]{geometry}
\usepackage{tikz}
\usepackage{hyperref}


\usetikzlibrary{arrows,decorations,calc}
\usetikzlibrary{decorations.pathmorphing,patterns,decorations.pathreplacing,decorations.markings}

\usepgflibrary{arrows}

\tikzset{
  % style to apply some styles to each segment of a path
  on each segment/.style={
    decorate,
    decoration={
      show path construction,
      moveto code={},
      lineto code={
        \path [#1]
        (\tikzinputsegmentfirst) -- (\tikzinputsegmentlast);
      },
      curveto code={
        \path [#1] (\tikzinputsegmentfirst)
        .. controls
        (\tikzinputsegmentsupporta) and (\tikzinputsegmentsupportb)
        ..
        (\tikzinputsegmentlast);
      },
      closepath code={
        \path [#1]
        (\tikzinputsegmentfirst) -- (\tikzinputsegmentlast);
      },
    },
  },
  % style to add an arrow in the middle of a path
  mid arrow/.style={postaction={decorate,decoration={
        markings,
        mark=at position .55 with {\arrow[#1]{stealth}}
      }}},
}




\newcommand{\diff}{\text{d}}



\begin{document}
\noindent
{\textsc{Universit\'e catholique de Louvain}} \hfill \'Ecole de Physique\\
Facult\'e des Sciences \hfill 23 septembre 2021\\
\hrule

\bigskip

\begin{center}
  \textbf{LPHYS2114 Introduction to non-linear dynamics}\\
  \textbf{Exercise 1 - One dimensional ODEs}
\end{center}

%\bigskip
\SetupExSheets{headings=runin-fixed-nr}

\subsubsection*{ODE properties}
\begin{question}
\textbf{The flow of a differential equation.} Determine a one dimensional ODE which has the flow diagram given by:
\begin{equation}
\begin{tikzpicture}[>=stealth,baseline]
   \draw[postaction={on each segment={mid arrow}}] (0,0) -- (1,0)--(2,0);
   \draw[postaction={on each segment={mid arrow}}] (4,0) -- (3,0)--(2,0);
   \draw[postaction={on each segment={mid arrow}}] (4,0) -- (5,0);
   \fill (2,0) circle (2pt);
   \draw[dotted] (5,0) -- (5.5,0);
    \draw[dotted] (-.5,0) -- (0,0);

   \draw[fill=white] (4,0) circle (2pt);
   \draw[fill=white] (1,0) circle (2pt);
   \draw[fill=black] (1,2pt) arc [start angle=90, end angle=270, radius=2pt];
   \draw (1,-.4) node {$-1$};
   \draw (2,-.4) node {$0$};
   \draw (4,-.4) node {$2$};
\end{tikzpicture}
\end{equation}

\end{question}

\begin{question}
\textbf{Finding ODEs with given properties.}
For each case below, find an example of an ODE $\dot x =  f(x)$ that has the corresponding property. If no example exists, justify why. Assume that in all cases the function $f(x)\in C^1$.
  \begin{enumerate}[label=(\alph*)]
    \item All real numbers are equilibria.
    \item All integers, but only integers, are equilibria.
    \item There are exactly three equilibria.
    \item There are exactly three equilibria, and all three are stable.
    \item There are no equilibria.
  \end{enumerate}
\end{question}


\begin{question}
\textbf{Gradient systems.} The aim of this exercise is to prove that one dimensional ODEs do not have periodic solutions. To do this we will use a more general result about \textit{gradient systems}. A gradient system is an ODE with the form:
\begin{equation}
  \dot{\bm x} = -\bm \nabla V(\bm x), \quad \bm x\in \mathbb R^n,
  \label{eqn:ODE}
\end{equation}
where $V: \mathbb R^n \to \mathbb R$  is a function of class $C^2$, which we call the potential. A periodic solution exists there exists a function $\bm x = \bm x(t)$ of class $C^1$ and there exists a time $T>0$ minimal, where $\bm x(t+T) = \bm x(t)$.
\begin{enumerate}[label=(\alph*)]
  \item Assume that $\bm x(t)$ is a solution of the differential equation in \eqref{eqn:ODE}. Show that:
  \begin{equation}
    V(\bm x(t)) = V(\bm x(0))-\int_0^t ||\bm\nabla V(\bm x(s))||^2\diff s
    \label{eqn:VEvolution}
  \end{equation}
  where $||\cdot ||$ denotes the Euclidean norm $\mathbb R^n$.

  \item Prove that \eqref{eqn:ODE} does not have periodic solutions.
    \item \textbf{Application 1 :} Show that any one dimensional differential equation $\dot x = f(x)$, $f$ of class $C^1$, can be written as a gradient system, and conclude about the existence of periodic solutions.
  \item \textbf{Application 2 :} Does the following system permit periodic solutions? 
  \begin{equation}
    \dot x_1 =x_2+2x_1x_2, \quad \dot x_2 = x_1 - x_2+x_1^2.
  \end{equation}
  \end{enumerate}

\end{question}

\subsubsection*{Applications}
\begin{question} \textbf{A leaky bucket.}
There is a bucket of water with a hole at the bottom. Let $h(t)$ be the height of the remaining water in the bucket at time $t$. Let $a$ be the area of the hole and $A$ be the area of a horizontal cross-section of the bucket (assume constant). Lastly, let $v(t)$ be the speed of the water passing through the hole.

\begin{enumerate}[label=(\alph*)]
\item Using conservation of mass, show that $av(t) = A \dot h(t)$.
\item Using conservation of energy, show that $v(t)^2=2gh(t)$. Derive the equation:
\begin{equation}
   \dot h(t) = -{a\over A} \sqrt{2g h(t)}.
\end{equation}
\item Given the initial condition $h(0)=0$ (meaning the bucket is empty at time $t=0$), show that the solution is not unique for $t<0$. Is this result intuitive? Why does the theorem of uniqueness of solutions not apply in this case?
\end{enumerate}
\end{question}


\begin{question}
\textbf{Population growth model.} We want to model the growth of a population of organisms. Say that $N=N(t)$ is the number of organisms at time $t$. For certain species the rate of growth is maximal for intermediate values of $N$. This could be because a low population number it could be difficult to find a partner, and competition for food increases when $N$ is large.
\begin{enumerate}[label=(\alph*)]
\item Show that the ODE
\begin{equation}
  \frac{\dot N}{N} = r-a(N-b)^2
\end{equation}
provides a model that satisfies these properties, provided the constants $r,\ a$ and $b$ satisfy certain constraints which you should determine.
\item Find the equilibria of the system and classify their stability.
\item Sketch the shape of the solutions for different initial conditions.
\item Compare the solutions $N(t)$ against those of the logistic equation. Are there qualitative differences?
\end{enumerate}
\end{question}

 
\end{document}

