%
%
%
%

\documentclass{article}
         
\usepackage{graphicx,xcolor}
\usepackage[load-headings]{exsheets}
\usepackage{bbding, skull}
\usepackage[frenchb]{babel}
\usepackage{amsmath,amssymb,bm}
\usepackage{enumitem}
\usepackage{lmodern,microtype}
\usepackage[a4paper,left=4cm,right=4cm]{geometry}
\usepackage{tikz}
\usepackage{hyperref}


\usetikzlibrary{arrows,decorations,calc}
\usetikzlibrary{decorations.pathmorphing,patterns,decorations.pathreplacing,decorations.markings}

\usepgflibrary{arrows}

\tikzset{
  % style to apply some styles to each segment of a path
  on each segment/.style={
    decorate,
    decoration={
      show path construction,
      moveto code={},
      lineto code={
        \path [#1]
        (\tikzinputsegmentfirst) -- (\tikzinputsegmentlast);
      },
      curveto code={
        \path [#1] (\tikzinputsegmentfirst)
        .. controls
        (\tikzinputsegmentsupporta) and (\tikzinputsegmentsupportb)
        ..
        (\tikzinputsegmentlast);
      },
      closepath code={
        \path [#1]
        (\tikzinputsegmentfirst) -- (\tikzinputsegmentlast);
      },
    },
  },
  % style to add an arrow in the middle of a path
  mid arrow/.style={postaction={decorate,decoration={
        markings,
        mark=at position .55 with {\arrow[#1]{stealth}}
      }}},
}



\renewcommand{\i}{\mathrm{i}}
\newcommand{\diff}{\text{d}}



\begin{document}
\noindent
{\textsc{Universit\'e catholique de Louvain}} \hfill \'Ecole de Physique\\
Facult\'e des Sciences \hfill 24 October 2024\\
\hrule

\bigskip

\begin{center}
  \textbf{LPHYS2114 Non-linear dynamics}\\
  \textbf{S\'erie 4 -- Vari\'et\'e stable-unstable and Lyapunov Functions} 
\end{center}

%\bigskip
\SetupExSheets{headings=runin-fixed-nr}

\begin{question}
  \textbf{Phase portrait of two dimensional systems.} Given the two dimensional system
  \begin{equation}
    \dot x = - x(1+y), \quad \dot y = y+x^2.
    \label{eqn:2DSystem}
  \end{equation}
  \begin{enumerate}[label=(\alph*)]
  
%%%%%%%%%%%%%%%%%%%%%%%%%%%%%%%%%%%%%%%%%%%%%%%%%%%%%%%%%%%%%%%%%%%%%%%%%%%%%%%%%%%%%
    \item Show that the system is invariant under the transformation $(x,y)\mapsto (-x,y)$. What does this tell us about the phase space of the system?
%%%%%%%%%%%%%%%%%%%%%%%%%%%%%%%%%%%%%%%%%%%%%%%%%%%%%%%%%%%%%%%%%%%%%%%%%%%%%%%%%%%%%

%%%%%%%%%%%%%%%%%%%%%%%%%%%%%%%%%%%%%%%%%%%%%%%%%%%%%%%%%%%%%%%%%%%%%%%%%%%%%%%%%%%%%
    \item Find and classify all the equilibria of the system. Qualitatively describe the dynamics in the neighbourhood of these equilibria.
%%%%%%%%%%%%%%%%%%%%%%%%%%%%%%%%%%%%%%%%%%%%%%%%%%%%%%%%%%%%%%%%%%%%%%%%%%%%%%%%%%%%%
    
%%%%%%%%%%%%%%%%%%%%%%%%%%%%%%%%%%%%%%%%%%%%%%%%%%%%%%%%%%%%%%%%%%%%%%%%%%%%%%%%%%%%%
    \item Approximately calculate the stable and unstable local manifolds of the saddle point equilibria by developing the Taylor series up to order $4$. \textit{Hint:} Use the symmetry of the system found in part (a) to simplify the calculation.
%%%%%%%%%%%%%%%%%%%%%%%%%%%%%%%%%%%%%%%%%%%%%%%%%%%%%%%%%%%%%%%%%%%%%%%%%%%%%%%%%%%%%

%%%%%%%%%%%%%%%%%%%%%%%%%%%%%%%%%%%%%%%%%%%%%%%%%%%%%%%%%%%%%%%%%%%%%%%%%%%%%%%%%%%%%
    \item Sketch the dynamics of the system.  
%%%%%%%%%%%%%%%%%%%%%%%%%%%%%%%%%%%%%%%%%%%%%%%%%%%%%%%%%%%%%%%%%%%%%%%%%%%%%%%%%%%%%
\end{enumerate}
\end{question}

\begin{question} \textbf{Stable and unstable manifold.} Given the system of ODEs
\begin{equation}
  \dot x = x(4-x-y), \quad \dot y = y(x-2).
\end{equation}
\begin{enumerate}[label=(\alph*)]
  \item Show that the system has saddle point two equilibria at $\bm p=(0,0)$ and $\bm p = (4,0)$, and a single attractor at $\bm p=(2,2)$.
  \item Determine the unstable and stable manifolds in the neighbourhood of the saddle points by developing the Taylor series up to $3$.
  
  \item Sketch the phase portrait of the system.
\end{enumerate}

\end{question}

\begin{question} \textbf{A Lyapunov function for the Lorenz 63 system.} The famous Lorenz 1963 system of ODEs is given by:
\begin{equation}
  \dot x = \sigma (y-x), \quad \dot y = r x - y - xz,\quad \dot z = xy-b z.
\end{equation}
Where, $\sigma,r,b>0$ are parameters. In this exercise we are interested in one of the equilibria:
\begin{enumerate}[label=(\alph*)]
  \item Show that $\bm p =(0,0,0)$ is an equilibria of the system for all $\sigma,r,b$.
  \item Show that $E(x,y,z)=\sigma^{-1}x^2+y^2+z^2$ is a strict Lyapunov function for this equilibria for $0< r < 1$.
  \item Describe the basin of attraction of this equilibria. (The basin of attraction of a equilibria is a collection of points that converge to the equilibria under the flow.)
\end{enumerate}
\end{question}

\begin{question}
 \textbf{Ideal pendulum with friction.} THe ideal pendulum, is a rigid pendulum of length $\ell$ under the influence of gravity and a friction force such as air pressure. The equation of motion is given by:
 \begin{equation}
   \ddot \theta + 2\gamma \dot \theta + \omega^2 \sin \theta = 0.
 \end{equation}
Here, $\gamma>0$ is the coefficient of friction and $\omega = \sqrt{g/\ell}$.
\begin{figure}[h]
  \centering
  \begin{tikzpicture}[>=stealth]
    \draw (-2,0)--(2,0);
    \draw (0,2) -- (0,-2);
    \draw (0,0) circle (1.5);
   
    \draw[rotate=-90] (1.,0) arc [start angle=0,end angle = 40, radius=1.cm];
    \draw (0.2,-.65) node {$\theta$};
   
    \draw[->] (2,-.5) -- (2,-1.5);
    \draw (2,-1) node [right] {$\bm g$};

    
    \begin{scope}[rotate=40]
    \draw[densely dashed] (0,0) -- (0,-1.5);
    \fill (0,-1.5) circle (1.5pt);
    \end{scope}
    \draw (.45,-.45) node[right] {$\ell$};

  \end{tikzpicture}
  \caption{Sketch of the idealised pendulum.}
\end{figure}

 \begin{enumerate}[label=(\alph*)]
   \item Write the equations of movement as a two dimensional system where $x_1 = \theta$ and $x_2 =\dot \theta$.
    \item Show that the equilibria of the system are of the form $\bm p = (n\pi,0),\,n\in \mathbb Z$. Why is it sufficient to consider only the points described by $n=0$ et $n=1$. Distinguish the physical meaning of these two equilibria.
    \item Show that the energy function $E = \frac12 x_2^2 + \omega^2(1-\cos x_1)$ is a Lyapunov function of the equilibria $\bm p=(0,0)$. Deduce that this equilibria is stable.
    \item Can we also deduce that $\bm p=(0,0)$ is asymptotically stable? Conclude about the dynamics of the system.
    \end{enumerate}
\end{question}

\begin{question}
  \textbf{Bendixson Criteria.} Show that the ODE $\ddot x = x - x^3 + (b-x^2)\dot x$ does not have periodic solutions for $b<0$.
  
\end{question}

 
\end{document}

