%
%
%
%

\documentclass{article}
         
\usepackage{graphicx,xcolor}
\usepackage[load-headings]{exsheets}
%\usepackage{bbding, skull}
\usepackage[frenchb]{babel}
\usepackage{amsmath,amssymb,amsthm,bm}
\usepackage{enumitem}
\usepackage{lmodern,microtype}
\usepackage[a4paper,left=4cm,right=4cm]{geometry}
\usepackage{tikz}
\usepackage{hyperref}



\newtheorem{theorem}{Theorem}[section]
\newtheorem{lemma}[theorem]{Lemma}
\newtheorem{proposition}[theorem]{Proposition}
\newtheorem{corollary}[theorem]{Corollary}
\newtheorem{conjecture}[theorem]{Conjecture}


\usetikzlibrary{arrows,decorations,calc}
\usetikzlibrary{decorations.pathmorphing,patterns,decorations.pathreplacing,decorations.markings}

\usepgflibrary{arrows}

\tikzset{
  % style to apply some styles to each segment of a path
  on each segment/.style={
    decorate,
    decoration={
      show path construction,
      moveto code={},
      lineto code={
        \path [#1]
        (\tikzinputsegmentfirst) -- (\tikzinputsegmentlast);
      },
      curveto code={
        \path [#1] (\tikzinputsegmentfirst)
        .. controls
        (\tikzinputsegmentsupporta) and (\tikzinputsegmentsupportb)
        ..
        (\tikzinputsegmentlast);
      },
      closepath code={
        \path [#1]
        (\tikzinputsegmentfirst) -- (\tikzinputsegmentlast);
      },
    },
  },
  % style to add an arrow in the middle of a path
  mid arrow/.style={postaction={decorate,decoration={
        markings,
        mark=at position .55 with {\arrow[#1]{stealth}}
      }}},
}



\renewcommand{\i}{\mathrm{i}}
\newcommand{\diff}{\text{d}}



\begin{document}
\noindent
{\textsc{Universit\'e catholique de Louvain}} \hfill \'Ecole de Physique\\
Facult\'e des Sciences \hfill 21 November 2021\\
\hrule

\bigskip

\begin{center}
  \textbf{LPHYS2114 Non-linear Dynamics}\\
  \textbf{S\'erie 9 -- Chaotic Maps} 
\end{center}

%\bigskip
\SetupExSheets{headings=runin-fixed-nr}

\subsection*{One dimensional Maps}

\begin{question}
  $\bm{p x \mod 1}  $. Given $\Omega = [0,1[$ and $f:\Omega \to \Omega$ defined by $f(x) = px \mod 1$ where $p\geqslant 2$ is an integer.
  \begin{enumerate}[label=(\alph*)]
    \item Study the map map by using the $p$-adique representation for $x\in \Omega$:
    \begin{equation}
      x = .b_1b_2b_3\cdots = \sum_{n=1}^\infty \frac{b_n}{p^n}, \quad b_n\in \{0,1,\dots,p-1\}.
    \end{equation}
    \item Show that the map $f$ is chaotic.
  \end{enumerate}
\end{question}

\begin{question}
  \textbf{} $\bm{(x+\alpha) \mod 1}  $. We will look at the interval $\Omega = [0,1[$. To measure the distance between two points $x,y\in \Omega$, we use the metric $d(x,y) = \min(|x-y|,1-|x-y|)$. We can visualise $\Omega$ as a circle and $d$ as the length of an arc (\`a un facteur pr\`es).
  
  \medskip
  \noindent Given $f:\Omega \to \Omega$ the map defined by $f(x) = (x+\alpha) \mod 1$ where $0\leqslant \alpha < 1$ is a real number.
  \begin{enumerate}[label=(\alph*)]
    \item Show that if $\alpha$ is rational then the points in $\Omega$ are periodic. Deduce that there are no dense orbits in $\Omega$.
    \item Show that if $\alpha$ is irrational then all the points have a dense orbit. Show that $f$ does not have periodic points.
    \item Show that for all $\alpha$, the map $f$ does not have the property of sensitivity to initial conditions and conclude.
  \end{enumerate}
\end{question}

\subsection*{Maps in two dimensions}
\begin{question}
\textbf{Bakers Map.} Given $\Omega=[0,1[\times [0,1[$. We denote $\bm x = (x_1,x_2) \in \Omega$ are points. The map $f:\Omega\to \Omega$ of the Baker Map is given by $\bm x' = f(\bm x)$ with
\begin{equation}
  x'_1 = 2x_1 \mod 1, \quad x'_2 = \frac{1}{2}(x_2 +[2x_1]) 
\end{equation}
where $[\xi]$ is a est la partie enti\`ere de $\xi\in \mathbb R$. 
\medskip

\noindent
Geometrically, the map can be visualised in two steps: \textit{(i)} We transform $\Omega$ a rectangle of length $2$ and height $1/2$ by the transformation $(x_1,x_2)\to (2x_1,x_2/2)$. \textit{(ii)} We cut the rectangle vertically into two rectangles of length $1$ and height $1/2$ and combine the two halves to regather the original square $\Omega$. An illustration is given in Figure \ref{fig:BakersMap}.
\begin{figure}[h]
  \centering
  \begin{tikzpicture}[>=stealth,scale=.75]
    \draw[->] (2.75,0)--(4.25,0);
    \draw (3.5,0) node[above] {$f$};
    \draw[->] (0,-2.25)--(0,-2.75);
    \draw[->] (7,-2.75)--(7,-2.25);
    \draw (0,-2.5) node[right]{\textit{(i)}};
    \draw (7,-2.5) node[left]{\textit{(ii)}};


    \begin{scope}
      \draw[dotted] (-2,-2) rectangle (2,2);
      \fill[fill=blue!50!white!30] (0,0) circle [radius=1cm];
    \end{scope}
    
    \begin{scope}[xshift=3.5cm,yshift=-4cm]
      \fill[fill=blue!50!white!30] (0,0) ellipse [x radius=2cm,y radius = .5 cm];
 
      \draw[dotted] (-4,-1) rectangle (4,1);
      \draw[dotted](0,1)--(0,-1);
    \end{scope}
    
    \begin{scope}[xshift=7cm]
      \draw[dotted](-2,-2) rectangle (2,2);
      \draw[dotted](-2,0)--(2,0);

      \clip (-2,-2) rectangle (2,2);

      \fill[fill=blue!50!white!30] (-2,1) ellipse [x radius=2cm,y radius = .5 cm];
      \fill[fill=blue!50!white!30] (2,-1) ellipse [x radius=2cm,y radius = .5 cm];

    \end{scope}
  \end{tikzpicture}
    \caption{An image of a disk of radius $r=1/4$ centred on the square $\Omega$ under one transformation of the bakers map.}
    \label{fig:BakersMap}

\end{figure}

\begin{enumerate}[label=(\alph*)]
  \item Why is $f$ called the ``bakers map"?
\end{enumerate}

  
  \subsubsection*{Dyadic Representation}
 \noindent
  We are interested in the map defined by this method. As the map $2x \mod 1$ on the interval, it is convenient to focus on $f$ and use a dyadic representation. For $x_1=.b_0b_1b_2\cdots$ and $x_2 = .b_{-1}b_{-2}b_{-3}\cdots$ we write:
    \begin{equation}
       \bm x = (x_1,x_2) = \cdots b_{-3}b_{-2}b_{-1}.b_0b_1b_2\cdots
       \label{eqn:BinaryRepresentation}
    \end{equation}
    In the series Dans la suite on \'ecrira \'egalement $b_n = b_n(\bm x)$ for the binary digits of $\bm x$.
    \begin{enumerate}[label=(\alph*)]
      \item Find $(x'_1,x'_2)$ and deduce that the action of $f$ in this representation.
      \item Show that $f$ is invertible on $\Omega$ and show that the action of $f^{-1}$ in the representation \eqref{eqn:BinaryRepresentation}. And deduce $f^{-1}$ in cartesian coordinates.
    \end{enumerate}
    \subsubsection*{Norm}
    \noindent To show that $f$ is chaotic we must introduce a norm on the interval $\Omega$ which allows us to measure distances. We can recall that all on the plane all norms are equivalent. It is convenient to choose the Manhattan norm:
    \begin{equation}
      ||\bm x|| = |x_1| + |x_2| ,\quad \bm x = (x_1,x_2)\in \Omega.
    \end{equation}
   
   \begin{enumerate}[label=(\alph*),resume] 
     \item Given $\bm x,\bm x' \in \Omega$ and $N> 0$ an integer. Show that if $b_n(\bm x) = b_n(\bm x')$ for all $-N \leqslant n \leqslant N-1$ then $||\bm x - \bm x'|| \leqslant 2^{-(N-1)}$.
     \item Given $E$ a sub-set of $\Omega$. Show that $E$ is dense in $\Omega$ if for all $\bm x\in \Omega,\,N>0$ an integer, there exists $\bm x'\in E$ such that $b_n(\bm x) = b_n(\bm x')$ for all $-N \leqslant n \leqslant N-1$.
   \end{enumerate}
   
   \subsubsection*{Chaos}

   \begin{enumerate}[label=(\alph*),resume] 
    \item Show that the bakers map is chaotic.
   \end{enumerate}

\end{question}

 
\end{document}

