%& -shell-escape
\documentclass[fleqn]{mc}
\definecolor{auburn}{rgb}{0.43, 0.21, 0.1}
\hypersetup{colorlinks = true,anchorcolor=blue,linkcolor=auburn}
\newtagform{auburn}{\color{auburn}(}{)}
\usetagform{auburn}
\setlength{\mathindent}{0cm}
\usetikzlibrary{decorations.pathmorphing,patterns}
\usetikzlibrary{arrows}
\tikzstyle{axis}=[>=latex,thick]
\usepackage{tabularx}
\usepackage{amsfonts}
\usepackage{natbib}
\bibliographystyle{alpha}
\setlength\parskip{12pt}
\def\cross{\times}
\input{Version.tex}
\renewcommand{\vec}[1]{\boldsymbol{#1}}
\renewcommand{\citet}{\cite}
\newcommand{\steady}[1]{{#1}_{r}}
\newcommand{\rf}[1]{{{#1}_{r}}}
\newcommand{\Dt}[1]{\frac{\Dif #1}{\Dif t}}
\newcommand{\dt}[1]{\frac{\dif #1}{\dif t}}
\newcommand{\Dtg}[1]{\left.\frac{\Dif #1}{\Dif t}\right|_g}
%\newcommand{\pdt}[1]{\pd{#1}{t}}
\newcommand{\pdx}[1]{\pd{#1}{x}}
\def\vpot{\varphi}
\def\grad{\boldsymbol{\nabla}}
\def\gradh{\nabla_h}
\def\div{\grad\cdot}
\def\anom{\tilde}
\def\ampl{\hat}
\def\divh{\nabla_h\cdot}
\newcommand{\pdt}[1]{\pd{#1}{t}}
\newcommand{\pdtt}[1]{\pd[2]{#1}{t}}

\usepackage{titlesec}
\titleformat{\section} {\normalfont\sffamily}{\hspace{-2em}\makebox[2em]{\thesection.}}{0em}{} 
\titleformat{\subsection} {\normalfont\sffamily}{\hspace{-2em}\makebox[2em]{\thesubsection.}}{0em}{} 
\titleformat{\subsubsection} {\normalfont\sffamily}{\hspace{-2em}\makebox[2em]{\thesubsubsection.}}{0em}{} 

\usepackage{silence}
\WarningFilter{hyperref}{removing}

\graphicspath{{Figures/}}

%\usepackage{siunitx}
%\includeonly{Session_1_Elementary_Notions}
%\includeonly{Session_2_Gravity_Waves}
%\includeonly{Session_3_Poincare_and_Kelvin_Waves}
\begin{document}
% \raggedright
\tableofcontents
\texttt{Git version \gitbranch:\gitrevision \ of \gitdate}
\section{Motivation and structure}

\subsection{Foreword}
\input{00_foreword}

\subsection{Motivating example: ice sheet}
\input{01_icesheet}

\subsection{Evaluation}
\input{02_evaluation}


\section{Continuous Flows: existence and uniqueness}
\input{Existence}

\section{Vector fields on the line: bifurcations and normal forms}
\input{Line}

\section{Vector fields on the plane: stable and unstable manifolds}
\input{Plane}
\section{Trapping regions,  attracting sets and attractors}
\input{Liapunov}

\section{Periodic orbits}
\input{Periodic}
\bibliography{/Users/crucifix/Documents/BibDesk.bib}

 

\end{document}

